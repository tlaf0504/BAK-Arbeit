Sofern nicht anders angegeben werden 6-knotige quadratische finite Dreieckselemente zur Approximation der Geometrie verwendet.

\subsection{Beispiel 1: Idealer Plattenkondensator}
In diesem Beispiel wird ein idealer Plattenkondensator unter Vernachlässigung der Randeffekte berechnet. Für ein solches Problem existiert eine analytische Lösung mit einem linearen Potentialverlauf $V(x) = kx$und einem konstanten elektrischen Feld $\vec{E} = \frac{U}{d}$ mit $d$ als Plattenabstand. Die Steigung $k$ des Potentialverlauf ergibt sich aus 
\begin{equation}
k = \frac{V^+ - V^-}{d}
\end{equation}
mit $V^+$ als dem Potential an der 'oberen' und $V^-$ dem Potential an der 'unteren' Elektrode.

Für dieses Problem liefert der FEM-Algorithmus die exakte Lösung liefern, da ein quadratischer Approximationsansatz verwendet wird und der Potentialverlauf ein Linearer ist.



Folgende Parameter sind gegeben: 
\begin{itemize}
	\item $U = 10 \si{\volt}$
	\item $d = 1 \si{\milli\meter}$
	\item $\epsilon = \epsilon_0$
\end{itemize}

woraus sich ein Potentialverlauf von 
\begin{equation}
V(x) = \frac{10V - 0V}{1\si{\milli\meter}} = 10000\si{\volt\per\meter}\ x
\end{equation} mit $x = [0,d]$ und ein elektrisches Feld von 
\begin{equation}
\vec{E} = -10000 \si{\volt\per\meter}\vec{e_x}
\end{equation}
ergibt.\newline

\begin{figure}[htbp]
	\centering
	\includegraphics[width=\textwidth]{pics/example_1/geometry.eps}
	\caption{Geometrie, Gitter und Randbedingungen zu Beispiel 1}
\end{figure}

\textbf{Anmerkung:} Die homogenen neumannschen Randbedingungen $\parDiff{V}{n} = 0$ müssen nicht explizit angegeben werden.\newline


\begin{figure}[htbp]
	\begin{minipage}{0.5\textwidth}
		\includegraphics[width=\textwidth]{pics/example_1/figure_4.eps}
	\end{minipage}
	\begin{minipage}{0.5\textwidth}
		\includegraphics[width=\textwidth]{pics/example_1/figure_3.eps}
	\end{minipage}
\caption{Potentialverlauf des Plattenkondensators}
\label{fig:plate_cape_potential}
\end{figure}

\begin{figure}[htbp]
	\begin{minipage}{0.5\textwidth}
		\includegraphics[width=\textwidth]{pics/example_1/figure_5.eps}
	\end{minipage}
	\begin{minipage}{0.5\textwidth}
		\includegraphics[width=\textwidth]{pics/example_1/figure_6.eps}
	\end{minipage}
	\caption{Feldverlauf des Plattenkondensators}
	\label{fig:plate_cape_field}
\end{figure}

Abbildung \ref{fig:plate_cape_potential} zeigt den linearen Potentialverlauf innerhalb des Kondensators. Wie unschwer zu erkenne ist ist der Potentialverlauf ein linearer und stimmt mit der analytischen Lösung überein.\newline
 Abbildung \ref{fig:plate_cape_field} zeigt im linken Teil die (nicht skalierten) Feldlinien und im rechten Teil den Absolutwert des Feldes. Auch hier ist eine Übereinstimmung mit der analytischen Lösung erkennbar. Man beachte herzu die Skala rechts außen. Die Abweichung vom analytischen Wert von $10000\si{\volt\per\meter}$ liegt im Bereich der numerischen Ungenauigkeit.
\newpage


\subsection{Beispiel 2: Strömungsfeld einer Blechplatte}
Dieses Beispiel zeigt den Strömungsfeldverlauf einer Blechplatte und wurde aus \cite{SMS_VO_skript}, Kap. 7.2.1 übernommen.\newline

\begin{figure}[H]
	\centering
	\includegraphics[scale=0.28]{pics/example_2/geometry.eps}
	\caption{Geometrie, Gitter und Randbedingungen zu Beispiel 2}
\end{figure}



\begin{figure}[htbp]
	\begin{minipage}{0.5\textwidth}
		\includegraphics[width=\textwidth]{pics/example_2/figure_4.eps}
	\end{minipage}
	\begin{minipage}{0.5\textwidth}
		\includegraphics[width=\textwidth]{pics/example_2/figure_3.eps}
	\end{minipage}
	\caption{Potentialverlauf der Blechplatte}
	\label{fig:metal_plate_potential}
\end{figure}

Abbildung \ref{fig:metal_plate_potential} zeigt wieder den Verlauf des berechneten Potentials.\newline
Abbildung \ref{fig:metal_plate_field} zeigt den Feldverlauf des Strömungsfeldes. Das Strömungsfeld wurde über \begin{equation}
\vec{J} = -[\gamma] \mathit{grad}V \text{ mit } [\gamma] = \begin{bmatrix}1 & 0 \\ 0 & 1\end{bmatrix}
\end{equation}berechnet. Somit decken sich in diesem Beispiel die Feldverläufe von $\vec{J}$ und $\vec{E}$.\newline
Wie zu erwarten tritt an der Ecke des inneren Randes eine starke Konzentration des Feldes mit einem entsprechend hohen Absolutwert des Feldes auf. Bei der Generierung des Gitters ist hier auf eine entsprechende Auflösung zu achten.
\begin{figure}[htbp]
	\begin{minipage}{0.5\textwidth}
		\includegraphics[width=\textwidth]{pics/example_2/figure_5.eps}
	\end{minipage}
	\begin{minipage}{0.5\textwidth}
		\includegraphics[width=\textwidth]{pics/example_2/figure_6.eps}
	\end{minipage}
	\caption{Feldverlauf der Blechplatte}
	\label{fig:metal_plate_field}
\end{figure}


\subsection{Beispiel 3: Zylinderelektroden über leitender Ebene}
Dieses Beispiel zeigt den Feldverlauf zweier Zylinderelektroden über einer leitenden Ebene. Die Elektroden befinden sich $100 \si{\milli\meter}$ bzw. $200 \si{\milli\meter}$ in einem Abstand von $100 \si{\milli\meter}$ über der leitenden Ebene. Die Ebene besitzt ein Potential von $V = 0 \si{\volt}$

\begin{figure}[H]
	\begin{minipage}{0.5\textwidth}
		\includegraphics[width=\textwidth]{pics/example_3/geometry_full.eps}
	\end{minipage}
	\begin{minipage}{0.5\textwidth}
		\includegraphics[width=\textwidth]{pics/example_3/geometry.eps}
	\end{minipage}
	\caption{Geometrie, Gitter und Randbedingungen zu Beispiel 3}
	\label{fig:example_3_geometry}
\end{figure}

Abbildung \ref{fig:example_3_geometry} zeigt die Geometrie sowie das Gitter und die Randbedingungen des Problems. Im linken Teil erkennt man die gesamte Geometrie inklusive fernem Rand. Im rechten Teil ist der interessante Teil der Geometrie mit den Zylinderelektroden dargestellt. 

\begin{figure}[htbp]
	\begin{minipage}{0.5\textwidth}
		\includegraphics[width=\textwidth]{pics/example_3/figure_4.eps}
	\end{minipage}
	\begin{minipage}{0.5\textwidth}
		\includegraphics[width=\textwidth]{pics/example_3/figure_3_edited.eps}
	\end{minipage}
	\caption{Potentialverlauf der Zylinderelektroden}
	\label{fig:cylinder_electrodes_potential}
\end{figure}

Abbildung \ref{fig:cylinder_electrodes_potential} zeigt wie bereits bekannt den Potentialverlauf des Problems. Hier ist zu beachten dass beim rechten Teil die Ansicht um $180^{\circ}$ gedreht ist, was im linken Teil einer Blickrichtung von oben nach unten entspricht. Diese Ansicht wurde gewählt da sich sonst die Äquipotentialfläche der ersten Elektrode nur schwer zu erkennen ist.


\begin{figure}[htbp]
	\begin{minipage}{0.5\textwidth}
		\includegraphics[width=\textwidth]{pics/example_3/figure_5.eps}
	\end{minipage}
	\begin{minipage}{0.5\textwidth}
		\includegraphics[width=\textwidth]{pics/example_3/figure_6.eps}
	\end{minipage}
	\caption{Feldverlauf der Zylinderelektroden}
	\label{fig:cylinder_electrodes_field}
\end{figure}


Abbildung \ref{fig:cylinder_electrodes_field} zeigt den Feldverlauf und den Absolutwert des Elektrischen Feldes.
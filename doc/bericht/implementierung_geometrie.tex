\subsection{Erstellung der Geometrie und Generierung des Gitters mit Gmsh}
\subsubsection{Erstellung der Geometrie}
Die Erstellung der Geometrie sowie die Generierung des FEM-Gitters erfolgt mit der Open-Source Software Gmsh. \cite{gmsh_website}
Gmsh stellt zwei CAD-Kernel ('Built-In' und 'OpenCASCADE') zur Verfügung welche jedoch sehr ähnlich zu Bedienen sind. Auf Unterschiede wird entsprechend hingewiesen.\newline

Gmsh nutzt dabei eine eigene Skript-Sprache zur Erstellung der Geometrie, wobei die einzelnen Kommandos für den jeweiligen CAD-Kernel übersetzt werden. Eine Geometrie in Gmsh wird somit durch ein ASCII-codiertes File mit einer Sequenz von Kommandos repräsentiert. Die Erstellung der Geometrie erfolgt dabei in \textit{Bottom-Up}-Reihenfolge. Dabei werden zuerst Punkte im Raum festgelegt, welche dann durch Linien verbunden werden. Geschlossene Kurven aus mehreren aneinander grenzenden Linien bilden eine Fläche und geschlossene Oberflächen aus mehreren aneinander grenzenden Flächen ein Volumen. Für eine genauere Beschreibung des Ablaufs sei auf die sehr ausführliche Dokumentation auf der Gmsh-Homepage verwiesen (\cite{gmsh_website}).\newline

Eine besondere Rolle bei der Erstellung der Problemgeometrie nehmen die sogenannten \textit{Physical Groups} ein. Wie ihr Name schon sagt, sind dies Gruppen von Elementen (z.B. Linien) mit den gleichen physikalischen Eigenschaften. (z.B. dem gleichen Potential). \textit{Physical curves} sind Gruppen von Linien und werden typischerweise zur Modellierung des dirichletschen oder neumannschen Randes verwendet. \textit{Physical surfaces} sind Gruppen von Flächen welche typischerweise zur Modellierung von Arealen gleichen Materials oder gleicher Raumladungsdichte verwendet werden.
 
 \subsubsection{Generierung des Gitters}
 Gmsh erlaubt die Verwendung von verschiedenen Algorithmen zur Gittergenerierung. In der Standardeinstellung wählt Gmsh automatisch einen geeigneten Algorithmus aus, wobei sich dies als völlig zufriedenstellend herausgestellt hat. Wie schon bei der Geometrie, erfolgt die Generierung des Gitters 'Bottom-Up' wobei zuerst Linen, dann Flächen und schließlich Volumen bearbeitet werden. Ein simpler Klick auf \textit{Mesh $\rightarrow$ 2D} führt alle genötigten Schritte durch, wobei standardmäßig ein lineares, 3-knotiges Dreiecksgitter erzeugt wird. Zum Wechsel auf ein Gitter höherer Ordnung ist ein Kick auf \textit{Mesh $\rightarrow$ Set order<n>} nötig, wobei <n> für die gewünschte Gitterordnung steht. Um das bestehende Gitter zu verfeinern, muss lediglich auf \textit{Mesh $\rightarrow$ Refine by splitting} geklickt werden. Bei unpassender Aufteilung des verfeinerten Gitters empfiehlt sich ein erneuter Klick auf \textit{Mesh $\rightarrow$ 2D} wodurch sich die Aufteilung der Elemente wieder verbessern sollte.\newline
 
 Beim Export des Gitters ist folgender Ablauf zu befolgen:
 \begin{itemize}
 	\item \textit{File $\rightarrow$ Export}
 	\item Auswahl des Ordners und Dateinamens \textbf{mit der Endung .msh} (Gmsh erkennt des Typ der zu Exportierenden Datei anhand seiner Endung)
 	\item Auswahl der 'Version 2' unter den anschließend angezeigten Optionen
 \end{itemize}

\textbf{Anmerkung:} Ein simpler Export mittels \textit{File $\rightarrow$ Save mesh} ist nicht möglich, da Gmsh dann das Gitter in einer .msh-Datei der Version 4 abspeichert, das FEM-Tool jedoch nur Version 2 unterstützt.\newline

\textbf{Anmerkung:} Die Generierung des Gitters kann auch automatisiert im FEM-Tool erfolgen. Somit ist ein händischer Export nur nötig wenn spezielle Änderungen am Gitter von Hand vorgenommen werde müssen.
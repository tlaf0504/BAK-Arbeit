\subsubsection{Berechnung der Elementmatrix-Koeffizienten $k_{ij}$}

Da die Formfunktionen $N_i$ bei isoparametrischen finiten Elementen Funktionen von $\xi$ und $\eta$ sind, müssen die Integrationsvariablen aller Integrale substituiert werden.  \newline
Die partiellen Ableitungen der Formfunktionen, wie sie z.B. in (\ref{eq:k_ij_e_static}) benötigt werden, ergeben nach \cite{SMS_VO_skript}, S.72 aus
\begin{equation}
\begin{Bmatrix}
\parDiff{N}{\xi} \\[2mm]
\parDiff{N}{\eta}
\end{Bmatrix} = 
\underbrace{
\begin{bmatrix}
\sum\limits_{k}\parDiff{N_k}{\xi}x_k & \sum\limits_{k}\parDiff{N_k}{\xi}y_k \\[2mm]
\sum\limits_{k}\parDiff{N_k}{\eta}x_k & \sum\limits_{k}\parDiff{N_k}{\eta}y_k
\end{bmatrix}}_{\mat{J}}\cdot
\begin{Bmatrix}
\parDiff{N}{x} \\[2mm]
\parDiff{N}{y}
\end{Bmatrix}
\end{equation}
zu
\begin{equation}
\label{eq:partial_derivatives_subst}
\begin{Bmatrix}
\parDiff{N}{x} \\[2mm]
\parDiff{N}{y}
\end{Bmatrix} = 
\underbrace{
	\frac{1}{\mathit{det}(\mat{J})}\begin{bmatrix}
	 \sum\limits_{k}\parDiff{N_k}{\eta}y_k & -\sum\limits_{k}\parDiff{N_k}{\xi}y_k \\[2mm]
	-\sum\limits_{k}\parDiff{N_k}{\eta}x_k & \sum\limits_{k}\parDiff{N_k}{\xi}x_k
	\end{bmatrix}}_{\mat{J}^{-1}}
\cdot \left\{ \begin{matrix}
\parDiff{N}{\xi} \\[2mm]
\parDiff{N}{\eta}
\end{matrix} \right\}_{.}
\end{equation}


Das Flächenelement $dxdy$ in (\ref{eq:k_ij_e_static}) ergibt sich mit den Beziehungen aus \cite{SMS_VO_skript}, S.74 zu 
\begin{equation}
\label{eq:surface_element}
dxdy = \mathit{det} \mat{(J)} d\xi d\eta,
\end{equation}
wobei das Flächenintegral in beiden Koordinaten in den Intervallen $\xi \in [0,1], \ \eta \in [0, 1-\xi]$ durchzuführen ist. Somit ergibt sich der Elementmatrix-Koeffizient $k_{ij}$ zu 
\begin{align}
&k_{i,j} = \int\displaylimits_{\xi = 0}^{1} \int\displaylimits_{\eta = 0}^{1 - \xi} \bigg[\\
%
% x-derivatives
&\frac{\epsilon_x}{\mathit{det}(\mat{J})}\left(\parDiff{N_i}{\xi}\sum\limits_{k}\parDiff{N_k}{\eta}y_k - \parDiff{N_i}{\eta}\sum\limits_{k}\parDiff{N_k}{\xi}y_k\right)
%
\left(\parDiff{N_j}{\xi}\sum\limits_{k}\parDiff{N_k}{\eta}y_k - \parDiff{N_j}{\eta} -\sum\limits_{k}\parDiff{N_k}{\xi}y_k\right) + \nonumber\\ 
%
% y-derivatives
&\frac{\epsilon_y}{\mathit{det}(\mat{J})}\left(-\parDiff{N_i}{\xi}\sum\limits_{k}\parDiff{N_k}{\eta}x_k + \parDiff{N_i}{\eta} \sum\limits_{k}\parDiff{N_k}{\xi}x_k\right)
%
\left(-\parDiff{N_j}{\xi}\sum\limits_{k}\parDiff{N_k}{\eta}x_k + \parDiff{N_j}{\eta} \sum\limits_{k}\parDiff{N_k}{\xi}x_k\right) \nonumber\\ 
& \bigg]d\xi d\eta
\end{align}

\textbf{Anmerkung:} Man beachte dass das $\mathit{det}(\mat{J})$ des Flächenelements aus (\ref{eq:surface_element}) bereits gekürzt wurde.\newline
Die Berechnung der partiellen Ableitungen der Formfunktionen kann auf analytischem Wege vorweg erfolgen, da diese nur vom gewählten Elementtyp abhängen. Somit ist eine effiziente Berechnung der Terme $\sum\limits_{k}\parDiff{N_k}{\eta}y_k$ in Form von Skalarprodukten möglich.\newline


\subsubsection{Berechnung der Rechtsseiten-Elemente $r_j$}
Die Berechnung der Koeffizienten der 'rechten Seite' eines elektrostatischen Problems 
\begin{equation}
r_j = \int\displaylimits_{x} \int\displaylimits_{y} N_i \rho dx dy + \int\displaylimits_{\Gamma_N} N_i\sigma d\Gamma
\end{equation}
erfordert zu einen ein Flächenintegral (erster Summand) und ein Integral über den neumannschen Rand, welches im zweidimensionalen Fall zu einem Kurvenintegral entartet. \newline
Für das Flächenintegral lässt sich die Substitution der Integrationsvariablen sehr einfach durchführen, da keine partiellen Ableitungen des Formfunktionen vorkommen:
\begin{equation}
\int\displaylimits_{x} \int\displaylimits_{y} N_i \rho dx dy = \int\displaylimits_{\xi = 0}^{1} \int\displaylimits_{\eta = 0}^{1 - \xi} N_i \rho \mathit{det} (\mat{J}) d\xi d\eta.
\end{equation}
\newline

Für Kurvenintegral über den neumannschen Rand ändert sich der Integrand je nach dem über welche Dreiecksseite integriert wird. 
Das entartete Flächenintegral über den neumannschen Rand hat nun folgende Form:
\begin{equation}
	\int\displaylimits_{c}N_i(\xi, \eta)\sigma ds.
\end{equation}

Allgemein gilt (siehe \cite{SMS_VO_skript}, S. 74f.):
\begin{align}
\label{eq:subs}
\vec{d\xi} = dx\vec{e_x} + dy\vec{e_y} = \parDiff{x}{\xi}d\xi \vec{e_x} + \parDiff{x}{\xi} d\xi \vec{e_y},\\
\vec{d\eta} = dx\vec{e_x} + dy\vec{e_y} = \parDiff{x}{\eta}d\eta \vec{e_x} + \parDiff{x}{\eta} d\eta \vec{e_y} .\nonumber
\end{align}
Das infinitesimale Kurvenelement ergibt sich somit nun zu $\vec{ds} = \vec{d\xi} + \vec{d\eta}$ bzw. \begin{equation}
\label{eq:ds_abs}
ds = \left|\left| \vec{ds} \right|\right| = \sqrt{\left(\vec{d\xi} + \vec{d\eta}\right)^T\cdot \left( \vec{d\xi} + \vec{d\eta} \right)}.
\end{equation}\newline

Integriert man über die Dreiecksseite 1, so gilt $d\eta = 0$ und somit auch $\vec{d\eta} = \vec{0}$.
Somit gilt für $\vec{ds}$ bzw. $ds$ unter Verwendung von (\ref{eq:ansatz_isoparam}):
\begin{align}
	&\vec{ds} =  \vec{d\xi} \Rightarrow \nonumber \\
	&ds = \left|\left| \vec{d\xi} \right|\right| = \sqrt{ \left(\parDiff{x}{\xi}\right)^2 + \left(\parDiff{y}{\xi}\right)^2} d\xi = \sqrt{ \left( \sum\limits_{k}\parDiff{N_k}{\xi}x_k \right)^2  + \left( \sum\limits_{k}\parDiff{N_k}{\xi}y_k \right)^2 } d\xi
\end{align}

Selbiges kann für Dreiecksseite 2 unter Verwendung von $d\xi = 0$ hergeleitet werden. Es ergibt sich somit:
\begin{align}
&\vec{ds} =  \vec{d\eta} \Rightarrow \nonumber \\
&ds = \left|\left| \vec{d\eta} \right|\right| = \sqrt{ \left(\parDiff{x}{\eta}\right)^2 + \left(\parDiff{y}{\eta}\right)^2} d\eta = \sqrt{ \left( \sum\limits_{k}\parDiff{N_k}{\eta}x_k \right)^2  + \left( \sum\limits_{k}\parDiff{N_k}{\eta}y_k \right)^2 } d\eta
\end{align}

Zur Integration über Dreiecksseite 3  muss zuerst die Kurve parametrisiert werden:
\begin{align*}
&\vec{s(t)} = 
\begin{Bmatrix}
\xi(t) \\ \eta(t)
\end{Bmatrix} = 
\begin{Bmatrix}
1 - t \\ t
\end{Bmatrix} \Rightarrow\\
%
&\frac{ds}{dt} = 
\begin{Bmatrix}
\frac{d\xi}{dt} \\ \frac{d\eta}{dt}
\end{Bmatrix} = 
\begin{Bmatrix}
-1 \\ 1
\end{Bmatrix} \Rightarrow \\
% 
&\underline{\underline{d\xi = -dt, \ d\eta = dt}}
\end{align*}

Setzt man dies nun in (\ref{eq:subs}) und dies wiederum in (\ref{eq:ds_abs}) ein, so erhält man
\begin{equation}
ds = \sqrt{\left(\parDiff{x}{\eta} - \parDiff{x}{\xi}\right)^2  + \left(\parDiff{y}{\eta} - \parDiff{y}{\xi}\right)^2} dt
\end{equation}

Setzt man nun für $x$ und $y$ die entsprechenden Zusammenhänge aus (\ref{eq:ansatz_isoparam}) ein erhält man:
\begin{equation}
ds = \sqrt{  \left(  \sum\limits_{k}\parDiff{N_k}{\eta}x_k  - \sum\limits_{k}\parDiff{N_k}{\xi}x_k  \right)^2 + \left(  \sum\limits_{k}\parDiff{N_k}{\eta}y_k  - \sum\limits_{k}\parDiff{N_k}{\xi}y_k  \right)^2  } dt
\end{equation}
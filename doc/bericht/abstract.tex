
\section*{Zusammenfassung}
\vspace*{2cm}
\begin{large}

Die Methode der finiten Elemente ist eine der bekanntesten und am weitesten Verbreiteten Methoden zur Lösung von Randwertproblemen in unterschiedlichsten Disziplinen. Das Institut für Grundlagen und Theorie der Elektrotechnik hat zu diesem Zwecke vor vielen Jahren mit der Entwicklung einer kommerziell Verwendbaren Software für eben jene Probleme auf dem Bereich der Elektrotechnik entwickelt. Aus diesen Bemühungen sind die Softwarepakete EleFAnT3D und EleFAnt2D entstanden.\newline

Beim Einsatz in der Lehre geht der Fokus einer solchen Software jedoch weg von hoher Optimierung und großem Funktionsumfang hin zu einfacher Bedienung und leichter Erweiterbarkeit des Source-Codes. Aus diesen Überlegungen heraus wurde die Entwicklung einer, auf der Open-Source-Software Gmsh und dem weit verbreiteten Mathematik-Programm Matlab\textregistered, basierenden Software zur Lösung oben genannter Probleme zum Einsatz speziell in der Lehre gestartet.\newline

Diese Seminararbeit versteht sich als ein Teil vergangener, laufender und noch folgender Seminararbeiten zur sukzessiven Entwicklung dieser Software.  \newline
Ziel dieser Seminararbeit ist die Implementierung der Lösung von zweidimensionalen, ebenen elektrostatischen und stationären Strömungsfeld-Problemen, sowie die automatisierte Generierung des Gitters mittels der Software Gmsh, dessen Import und Verarbeitung in der Software.

\end{large}
Im vorherigen Abschnitt wurde näher auf die Berechnung der Elementgleichungssysteme eingegangen. Diese müssen nun zu einem 'großen' Gleichungssystem assembliert werden. Der Assemblierungsvorgang entspricht dabei einer simplen Addition der Gleichungssysteme (Siehe \cite{SMS_VO_Skript}, S.60ff.). Beispielhaft wird der Vorgang anhand von 2 Elementgleichungssystemen gezeigt.

Gleichungssystem 1:
\begin{align*}
k_{11}^1 V_1 + k_{12}^1 V_2 + k_{13}^1 V_3 = r_1^1\\
k_{21}^1 V_1 + k_{22}^1 V_2 + k_{23}^1 V_3 = r_2^1\\
k_{31}^1 V_1 + k_{32}^1 V_2 + k_{33}^1 V_3 = r_3^1
\end{align*}


Gleichungssystem 2:
\begin{align*}
k_{22}^2 V_2 + k_{23}^2 V_3 = r_2^2 - k_{21}^2 V_4\\
k_{32}^2 V_2 + k_{33}^2 V_3 = r_3^2 - k_{31}^2 V_4
\end{align*}

Addiert man die Gleichungssysteme so ergibt sich
\begin{align}
k_{11}^1 V_1 + k_{12}^1 V_2 + k_{13}^1 V_3 +  &= r_1^1\\
k_{21}^1 V_1 + (k_{22}^1 +  k_{22}^2) V_2 + (k_{23}^1 + k_{23}^2) V_3 &= r_2^1 + r_2^2 - k_{21}^2 V_4 \nonumber\\
k_{31}^1 V_1 + (k_{32}^1 + k_{32}^2)  V_2 + ( k_{33}^1 + k_{33}^2) V_3 &= r_3^1 + r_3^2 - k_{31}^2 V_4 \nonumber
\end{align}

Als Resultat des Assemblierungsvorgangs erhält man ein lineares Gleichungssystem mit $N - N_{dir}$ Unbekannten, wobei $N$ die Anzahl der Knoten und $N_{dir}$ die Anzahl der Knoten am dirichletschen Rand darstellt. Die Koeffizientenmatrix ist symmetrisch und schwach besetzt.
\newline

Die Lösung des Gleichungssystems erfolgt durch ein geeignetes Verfahren und liefert die Unbekannten Knotenpotentiale als Ergebnis. Je nach zugrundeliegendem Problemtyp kann anschließend die gesuchter Feldgröße berechnet werden. Für den Fall eines elektrostatischen Problems wäre dies
\begin{equation}
\vec{E} = -\mathit{grad}V = 
-\begin{Bmatrix}
\parDiff{V}{x} \\[2mm]
\parDiff{V}{y}
\end{Bmatrix} = 
%
-\begin{Bmatrix}
\sum_{k} \parDiff{N_k}{x} V_k\\[2mm]
\sum_{k} \parDiff{N_k}{y} V_k
\end{Bmatrix}
\end{equation}
mit 
\begin{align*}
\parDiff{N_k}{x} = \frac{1}{\mathit{det}(\mat{J})} \left( \parDiff{N_k}{\xi} \sum_{j} \parDiff{N_j}{\eta} y_j - \parDiff{N_k}{\eta} \sum_{j} \parDiff{N_j}{\xi} y_j \right) \\
\parDiff{N_k}{y} = \frac{1}{\mathit{det}(\mat{J})} \left( -\parDiff{N_k}{\xi} \sum_{j} \parDiff{N_j}{\eta} x_j + \parDiff{N_k}{\eta} \sum_{j} \parDiff{N_j}{\xi} x_j \right) \\
\end{align*}


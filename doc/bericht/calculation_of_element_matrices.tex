\subsection{Numerische Berechnung der Elementgleichungssysteme}
\label{sec:equation_system_calculation}

Wie aus Kapitel \ref{sec:problem_types} bereits bekannt, ergibt sich das Elementgleichungssystem in den Koordinaten $x$ und $y$ für ein elektrostatisches Problem als
\begin{equation}
\label{eq:k_ij_e_static_2}
k_{ij} = \int\displaylimits_{x} \int\displaylimits_{y} \left(\epsilon_x \parDiff{N_i}{x}\parDiff{N_j}{x} +  \epsilon_y\parDiff{N_i}{y}\parDiff{N_j}{y}\right)dx dy
\end{equation}

\begin{equation}
\label{eq:right_side_e_static_2}
r_j = \int\displaylimits_{x} \int\displaylimits_{y} N_i \rho dx dy + \int\displaylimits_{\Gamma_N} N_i\sigma d\Gamma
\end{equation} 
(siehe Kapitel \ref{sec:electrostatic_problems}). Die Berechnungen für stationäre Strömungsfeld-Probleme sind zu denen für elektrostatische Probleme äquivalent (Siehe Kapitel \ref{sec:stat_current_problems}) mit $\epsilon \rightarrow \gamma$, $\rho \rightarrow J_e$ und $\sigma \rightarrow 0$\newline

Die numerische Integration erfolgt über die sogenannte \textit{Gauss-Quadratur}, bei der das Integral mittels einer gewichteten Summe von Stützstellen des Intriganten approximiert wird.
\begin{equation}
\int\limits_{x_{min}}^{x_{max}} f(x) dx \approx \sum\limits_{k}w_kf(x_k)
\end{equation}
Eine Erweiterung auf mehrdimensionale Integrale ist durch Mehrfachsummen einfach möglich. Ein besonderes Augenmerk bei dieser Methode der numerischen Integration liegt dabei auf der Wahl der Stützstellen $x_k$ und der Gewichte $w_k$. 


Die Stützstellen und Gewichte zur Berechnung des Flächenintegrals für isoparametrsiche, dreieckige finite Elemente wurden aus \cite{bathe}, S.547. übernommen. Implementiert wurden eine 3-Punkte und eine 7-Punkte Integration.\newline

Die Stützstellen zur numerischen Kurvenintegration wurden aus \cite{bathe}, S.542 übernommen. Implementiert wurden Integrationsordnungen von $1$ bis $6$.


\subsubsection{Elementgleichungssysteme für Elemente am dirichletschen Rand}
Liegt das Element am dirichletschen Rand, so sind ein oder mehrere Knotenpotentiale bereits vorgegeben was zu einer Reduktion des Elementgleichungssystems führt. Gezeigt wird dies an einem linearen Dreieckselement mit 3 Knoten. Das Elementgleichungssystem lautet:

\begin{equation}
\begin{bmatrix}
k_{11} & k_{12} & k_{13} \\
k_{21} & k_{22} & k_{23} \\
k_{31} & k_{32} & k_{33} \\
\end{bmatrix} \cdot 
\begin{Bmatrix}
V_1 \\
V_2 \\
V_3
\end{Bmatrix} = 
\begin{Bmatrix}
r_1 \\
r_2 \\
r_3
\end{Bmatrix}
\end{equation}

Liegt nun Knoten 2 am dirichletschen Rand so ist $V_2$ bekannt, und das Elementgleichungssystem reduziert sich wie folgt:

\begin{equation}
\begin{bmatrix}
k_{11}  & k_{13} \\
k_{31}  & k_{33} \\
\end{bmatrix} \cdot 
\begin{Bmatrix}
V_1 \\
V_3
\end{Bmatrix} = 
\begin{Bmatrix}
r_1 - k_{12} V_2 \\
r_3 - k_{32} V_2
\end{Bmatrix}
\end{equation}

Liegt also der $j$-te Knoten am dirichletschen Rand, so wird die $j$-te Zeile aus dem Gleichungssystem eliminiert und die $j$-ze Spalte wird von der rechte Seite subtrahiert.



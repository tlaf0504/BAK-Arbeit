\section{Grundlagen Antennen}
In diesem Kapitel werden einige Kenndaten von Antennen und deren Ermittlung besprochen. Später werden zwei theoretische Antennenmodelle vorgestellt, welche oft zum Vergleich mit realen Antennen herangezogen werden. 
\subsection{Allgemeines}
Um eine Antenne zu untersuchen, muss zuerst die Feldverteilung anhand dessen Geometrie ermittelt werden. 
Da für komplexere Antennen keine geschlossene Lösung ermittelbar ist, geschieht dies numerisch. In unserem Fall mittels der institutseigenen Software EleFAnT3D. \newline
In unserer Betrachtung ist die Richtwirkung der Antenne von Bedeutung. Um diese zu untersuchen, wird aus der EleFAnT3D-Software der Maximalwert des Scheitelwerts der elektrischen Feldstärke, hier mit $\hat{E}\left(\vartheta,\varphi \right)$ bezeichnet, auf einer Kugeloberfläche mit gegebenem Radius ermittelt. Anhand dieser, lassen sich dann die nachfolgend beschriebenen Kenndaten ermitteln.
\subsection{Kenndaten}
Die folgenden Kenndaten werden mittels der elektrischen Feldstärke im Fernfeld der Antenne berechnet. Es wird davon ausgegangen, dass der Radius r so gegeben ist, dass man sich im Fernfeld der Antenne befindet. Die Formeln wurden alle der Fachliteratur entnommen. \citep[vgl.][]{balanis} \citep[vgl.][]{orfanidi} \citep[vgl.][]{kark}
\begin{itemize}


\item Die \textbf{Strahlungsintensität $U$} einer Antenne ist die abgestrahlte Leistung pro Raumwinkel in eine bestimmte  Raumrichtung. Die Einheit ist definiert mit $\left[ U \right]=\frac{W}{sr}$. Sie wird mit
\begin{align*}
U\left(\vartheta,\varphi \right) = \frac{\hat{E}\left(\vartheta,\varphi \right)^2r^2 }{2\eta}
\end{align*} 
berechnet, wobei $\eta$ der reelle Feldwellenwiderstand definiert als $\eta=\sqrt{\frac{\mu_0}{\epsilon_0}}$, ist.

\item Die \textbf{Strahlungsleistungsdichte $W$} einer Antenne ergibt sich aus der auf die Größe der Öffnung im strahlenden Hohlraum bezogenen Strahlungsleistung. Sie hat die Einheit $\left[W \right]=\frac{W}{m^2}$ und ist definiert mit:
\begin{align*}
W\left(\vartheta,\varphi \right) = \frac{U\left(\vartheta,\varphi \right) }{r^2}
\end{align*} 

\item Als \textbf{Richtcharakteristik} einer Antenne werden in der Literatur verschiedene Kennwerte beschrieben. Einerseits die Richtcharakteristik \textbf{$C$} welche die Winkelverteilung des elektrischen Fernfeldes, bezogen auf den Maximalwert in Hauptstrahlungsrichtung beschreibt. Sie ist dimensionslos und definiert mit:
\begin{align*}
C\left(\vartheta,\varphi \right) = \frac{\hat{E}\left(\vartheta,\varphi \right) }{\hat{E}\left(\vartheta,\varphi \right)_{max}}
\end{align*} 
\newline
Gebräuchlicher ist die Richtcharakteristik \textbf{$D$} (engl. Directivity), definiert als die Strahlungsintensität $U$, bezogen auf die des idealen isotropen Strahlers $U_0$, welcher im nächsten Kapitel besprochen wird. Sie ist ebenfalls dimensionslos und folgend definiert:
\begin{align*}
D\left(\vartheta,\varphi \right) = \frac{U\left(\vartheta,\varphi \right)}{U_0}
\end{align*}
\newline
Wird die Directivity ohne Richtungsinformation angegeben, ist die Richtung der maximalen Directivity gemeint: $D=D_{max}=\frac{U\left(\vartheta,\varphi \right)_{max}}{U_0}$
\item Für eine bessere Darstellung wird meist der logarithmierte Wert als Antennengewinn (engl. Gain) angegeben:
\begin{align*}
G\left(\vartheta,\varphi \right) =10log_{10}\left(D\left(\vartheta,\varphi \right)  \right)
\end{align*}

\end{itemize}



\subsection{Isotroper Kugelstrahler}
Der isotroper Kugelstrahler ist ein theoretisches Modell einer Antenne, welche isotrop, also gleichmäßig in alle Raumrichtungen, und verlustlos sendet und empfängt. Um eine beliebige Antenne sinnvoll zu vergleichen, wird als Leistung des isotropen Kugelstrahlers die Sendeleistung $P_{s}$ der zu vergleichenden Antenne angegeben. Die Strahlungsintensität kann folgendermaßen angegeben werden:
\begin{align}
U_{iso}\left(\vartheta,\varphi \right) =U_0=Wr^2=\frac{P_s}{4 \pi }
\end{align}

\subsection{Hertzscher Dipol}
Als Hertzschen Dipol kann man eine infinitesimal kleine Linearantenne verstehen. Er kann als elektrischer Dipolmoment $p$, welcher sinusförmig variiert, verstanden werden. Die Strahlungsintensität kann folgendermaßen angegeben werden:
\begin{align}
U=Wr^2=\frac{3P_s}{8\pi}sin^2\left( \theta \right)
\end{align}

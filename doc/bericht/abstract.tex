Die Methode der finiten Elemente ist eine der am weitesten verbreiteten Methoden zur numerischen Lösungen von Randwertproblemen in Wissenschaft und Technik. Am Institut für Grundlagen und Theorie der Elektrotechnik begann man daher schon vor vielen Jahren mit der Entwicklung von Softwarepaketen zur numerischen Lösung solcher Probleme aus dem Gebiet der Elektrotechnik. Aus diesen Bemühungen entstanden die Softwarepakete \textit{EleFAnT2D} und \textit{EleFAnT3D}.\newline
Obwohl über die Jahre stetig erweitert und optimiert, wurde in jüngster Vergangenheit der Entschluss gefasst ein neues Softwarepaket mit speziellem Fokus auf die Verwendung in der Lehre zu entwickeln, da die \textit{EleFAnT}-Softwarepakete durch ihre Mächtigkeit in ihrer Handhabung recht komplex sind.\newline
Diese Seminararbeit ist Teil der Entwicklung jenes neuen Programms und soll sich auf grundlegende Funktionen wie die Anbindung des CAD-Programms \textit{Gmsh}\cite{gmsh_website} zur Erstellung der Geometrie und des Gitters, sowie der Implementierung eines einfachen FEM-Solvers zur Lösung zweidimensionaler, ebener Elektrostatik- und stationärer Strömungsfeld-Probleme fokussieren.
\newpage
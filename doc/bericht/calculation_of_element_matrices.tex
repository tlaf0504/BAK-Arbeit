\subsection{Berechnung der Elementgleichungssysteme}
\label{sec:equation_system_calculation}

Das Elementgleichungssystem in den Koordinaten $x$ und $y$ ergibt sich für ein elektrostatisches Problem als
\begin{equation}
\label{eq:k_ij_e_static_2}
k_{ij} = \int\displaylimits_{x} \int\displaylimits_{y} \left(\epsilon_x \parDiff{N_i}{x}\parDiff{N_j}{x} +  \epsilon_y\parDiff{N_i}{y}\parDiff{N_j}{y}\right)dx dy
\end{equation}

\begin{equation}
\label{eq:right_side_e_static_2}
r_j = \int\displaylimits_{x} \int\displaylimits_{y} N_i \rho dx dy + \int\displaylimits_{\Gamma_N} N_i\sigma d\Gamma
\end{equation} 
(siehe Kapitel \ref{sec:electrostatic_problems}). Die Berechnungen für stationäre Strömungsfeld-Probleme sind zu denen für elektrostatische Probleme äquivalent (Siehe Kapitel \ref{sec:stat_current_problems}) für $\epsilon \rightarrow \gamma$, $\rho \rightarrow J_e$ und $\sigma \rightarrow 0$\newline

Liegt das Element am dirichletschen Rand, so sind ein oder mehrere Knotenpotentiale bereits vorgegeben was zu einer Reduktion des Elementgleichungssystems führt. Gezeigt wird dies an einem linearen Dreieckselement mit 3 Knoten. Das Elementgleichungssystem lautet:

\begin{align*}
	k_{11} V_1 + k_{12} V_2 + k_{13} V_3 = r_1\\
	k_{21} V_1 + k_{22} V_2 + k_{23} V_3 = r_2\\
	k_{31} V_1 + k_{32} V_2 + k_{33} V_3 = r_3
\end{align*}

Liegt nun Knoten 2 am dirichletschen Rand so ist $V_2$ bekannt, und das Elementgleichungssystem reduziert sich wie folgt:

 \begin{align*}
 k_{11} V_1  + k_{13} V_3 = r_1 - k_{12} V_2\\
 k_{31} V_1  + k_{33} V_3 = r_3 - k_{32} V_2
 \end{align*}
 
 Liegt also der $j$-te Knoten am dirichletschen Rand, so wird die $j$-te Zeile aus dem Gleichungssystem eliminiert und die $j$-ze Spalte wird von der rechte Seite subtrahiert.


\subsubsection{Elementmatrix-Koeffizienten $k_{ij}$}

Da die Formfunktionen $N_i$ bei isoparametrischen finiten Elementen Funktionen von $\xi$ und $\eta$ sind, müssen die Integrationsvariablen aller Integrale transformiert werden.  \newline
Die partiellen Ableitungen der Formfunktionen in (\ref{eq:k_ij_e_static_2}) ergeben sich mit 
\begin{equation}
\begin{Bmatrix}
\parDiff{N}{\xi} \\[2mm]
\parDiff{N}{\eta}
\end{Bmatrix} = 
\underbrace{
\begin{bmatrix}
\sum\limits_{k}\parDiff{N_k}{\xi}x_k & \sum\limits_{k}\parDiff{N_k}{\xi}y_k \\[2mm]
\sum\limits_{k}\parDiff{N_k}{\eta}x_k & \sum\limits_{k}\parDiff{N_k}{\eta}y_k
\end{bmatrix}}_{\mat{J}}\cdot
\begin{Bmatrix}
\parDiff{N}{x} \\[2mm]
\parDiff{N}{y}
\end{Bmatrix}
\end{equation}

zu 
\begin{equation}
\begin{Bmatrix}
\parDiff{N}{x} \\[2mm]
\parDiff{N}{y}
\end{Bmatrix} = 
\underbrace{
	\frac{1}{\mathit{det}(\mat{J})}\begin{bmatrix}
	 \sum\limits_{k}\parDiff{N_k}{\eta}y_k & -\sum\limits_{k}\parDiff{N_k}{\xi}y_k \\[2mm]
	-\sum\limits_{k}\parDiff{N_k}{\eta}x_k & \sum\limits_{k}\parDiff{N_k}{\xi}x_k
	\end{bmatrix}}_{\mat{J}^{-1}}
\cdot \begin{Bmatrix}
\parDiff{N}{\xi} \\[2mm]
\parDiff{N}{\eta}
\end{Bmatrix}
\end{equation}\cite{SMS_VO_skript}
, S.71f.\newline


Das Flächenelement $dxdy$ in (\ref{eq:k_ij_e_static_2}) ergibt sich mit den Beziehungen aus \cite{SMS_VO_skript}, S.74 zu 
\begin{equation}
\label{eq:surface_element}
dxdy = \mathit{det} \mat{(J)} d\xi d\eta
\end{equation}
wobei das Flächenintegral in beiden Koordinaten in den Intervallen $[0,1]$ durchzuführen ist. Somit ergibt sich der Elementmatrix-Koeffizient $k_{ij}$ zu 
\begin{align}
&k_{i,j} = \int\displaylimits_{\xi = 0}^{1} \int\displaylimits_{\eta = 0}^{1 - \xi} \bigg[\\
%
% x-derivatives
&\frac{\epsilon_x}{\mathit{det}(\mat{J})}\left(\parDiff{N_i}{\xi}\sum\limits_{k}\parDiff{N_k}{\eta}y_k - \parDiff{N_i}{\eta}\sum\limits_{k}\parDiff{N_k}{\xi}y_k\right)
%
\left(\parDiff{N_j}{\xi}\sum\limits_{k}\parDiff{N_k}{\eta}y_k - \parDiff{N_j}{\eta} -\sum\limits_{k}\parDiff{N_k}{\xi}y_k\right) + \nonumber\\ 
%
% y-derivatives
&\frac{\epsilon_y}{\mathit{det}(\mat{J})}\left(-\parDiff{N_i}{\xi}\sum\limits_{k}\parDiff{N_k}{\eta}x_k + \parDiff{N_i}{\eta} \sum\limits_{k}\parDiff{N_k}{\xi}x_k\right)
%
\left(-\parDiff{N_j}{\xi}\sum\limits_{k}\parDiff{N_k}{\eta}x_k + \parDiff{N_j}{\eta} \sum\limits_{k}\parDiff{N_k}{\xi}x_k\right) \nonumber\\ 
& \bigg]d\xi d\eta
\end{align}

\textbf{Anmerkung:} Man beachte dass das $\mathit{det}(\mat{J})$ des Flächenelements aus (\ref{eq:surface_element}) bereits gekürzt wurde.\newline
Die Berechnung der Partiellen Ableitungen der Formfunktionen kann auf analytischem Wege vorweg erfolgen, da diese nur vom gewählten Elementtyp abhängen. Somit ist eine effiziente Berechnung der Terme $\sum\limits_{k}\parDiff{N_k}{\eta}y_k$ in Form von Skalarprodukten möglich.\newline

Die numerische Berechnung des Integrals erfolgt über die sogenannte \textit{Gauss-Quadratur}. Hierbei wird das zu berechnende Integral durch eine gewichtete Summe approximiert:
\begin{equation}
\int\limits_{x_{min}}^{x_{max}} f(x) dx \approx \sum\limits_{k}w_kf(x_k)
\end{equation}
Eine Erweiterung auf mehrdimensionale Integrale ist durch Mehrfachsummen einfach möglich. Ein besonderes Augenmerk bei dieser Methode der numerischen Integration liegt dabei auf der Wahl der Stützstellen $x_k$ und der Gewichte $w_k$. Diese Form der numerischen Integration ist in der Literatur weit verbreitet. Die in dieser Arbeit verwendeten Stützstellen-Koordinaten und -Gewichte für isoparametrische Dreieckige finite Elemente wurden aus \cite{bathe}, S.547. übernommen. \newline

In der Software implementiert sind eine 3-Punkte und 7-Punkte Integration der Form  
\begin{equation}
\int\displaylimits_{x} \int\displaylimits_{y} f dx dy \approx \frac{1}{2}\sum\limits_{k = 1}^{N_s} f(x_k, y_k) w_k
\end{equation}
wobei $N_s$ die Anzahl der Stützstellen darstellt.\newline

Für eine 3-Punkte Integration ergeben sich folgende Stützstellen und Gewichte:
\begin{table}[H]
	\centering
\begin{tabular}{c|c|c|c}
	$\mathbf{k}$ & $\mathbf{x_k}$ & $\mathbf{y_k}$ & $\mathbf{w_k}$ \\
	\hline
	1 & $\frac{1}{6}$ & $\frac{1}{6}$ & $\frac{1}{3}$ \\
	2 & $\frac{2}{3}$ & $\frac{1}{6}$ & $\frac{1}{3}$ \\
	3 & $\frac{1}{6}$ & $\frac{2}{3}$ & $\frac{1}{3}$ \\
\end{tabular}
\caption{Stützstellen und Gewichte für 3-Punkte Integration nach \cite{bathe}, S.547.}
\label{tab:3-point_plane_integration}
\end{table}

Für eine 7-Punkte Integration ergeben sich folgende Stützstellen und Gewichte:
\begin{table}[H]
	\centering
	\begin{tabular}{c|c|c|c}
		$\mathbf{k}$ & $\mathbf{x_k}$ & $\mathbf{y_k}$ & $\mathbf{w_k}$ \\
		\hline
		1 & $0.1012865073235$ & $0.1012865073235$ & $0.1259391805448$ \\
		2 & $0.7974269853531$ & $0.1012865073235$ & $0.1259391805448$ \\
		3 & $0.1012865073235$ & $0.7974269853531$ & $0.1259391805448$ \\
		4 & $0.4701420641051$ & $0.0597158717898$ & $0.1323941527885$ \\
		5 & $0.4701420641051$ & $0.4701420641051$ & $0.1323941527885$ \\
		6 & $0.0597158717898$ & $0.4701420641051$ & $0.1323941527885$ \\
		7 & $\frac{1}{3}$ & $\frac{1}{3}$ & $0.225$ \\
	\end{tabular}
	\caption{Stützstellen und Gewichte für 7-Punkte Integration nach \cite{bathe}, S.547.}
	\label{tab:7-point_plane_integration}
\end{table}



\subsubsection{Rechtsseiten-Elemente $r_j$}
Die Berechnung der Koeffizienten der 'rechten Seite' eines elektrostatischen Problems 
\begin{equation}
r_j = \int\displaylimits_{x} \int\displaylimits_{y} N_i \rho dx dy + \int\displaylimits_{\Gamma_N} N_i\sigma d\Gamma
\end{equation}$r_j$
erfordert zu einen ein Flächenintegral (erster Summand) und ein Integral über den neumannschen Rand, welches im zweidimensionalen Fall zu einem Kurvenintegral entartet. \newline
Für das Flächenintegral lässt sich die Substitution der Integrationsvariablen sehr einfach durchführen, da keine partiellen Ableitungen des Formfunktionen vorkommen:
\begin{equation}
\int\displaylimits_{x} \int\displaylimits_{y} N_i \rho dx dy = \int\displaylimits_{\xi = 0}^{1} \int\displaylimits_{\eta = 0}^{1 - \xi} N_i \rho \mathit{det} (\mat{J}) d\xi d\eta
\end{equation}
\newline

Für Kurvenintegral über den neumannschen Rand ändert sich der Integrand je nach dem über welche Dreiecksseite integriert wird. 
Das entartete Flächenintegral über den neumannschen Rand hat nun folgende Form:
\begin{equation}
	\int\displaylimits_{c}N_i(\xi, \eta)\sigma ds
\end{equation}

Allgemein gilt (siehe \cite{SMS_VO_skript}, S. 74f.):
\begin{align}
\label{eq:subs}
\vec{d\xi} = dx\vec{e_x} + dy\vec{e_y} = \parDiff{x}{\xi}d\xi \vec{e_x} + \parDiff{x}{\xi} d\xi \vec{e_y}\\
\vec{d\eta} = dx\vec{e_x} + dy\vec{e_y} = \parDiff{x}{\eta}d\eta \vec{e_x} + \parDiff{x}{\eta} d\eta \vec{e_y} \nonumber
\end{align}
Das infinitesimale Kurvenelement ergibt sich somit nun zu $\vec{ds} = \vec{d\xi} + \vec{d\eta}$ bzw. \begin{equation}
\label{eq:ds_abs}
ds = \left|\left| \vec{ds} \right|\right| = \sqrt{\left(\vec{d\xi} + \vec{d\eta}\right)^T\cdot \left( \vec{d\xi} + \vec{d\eta} \right)}
\end{equation}\newline

Integriert man über die Dreiecksseite 1, so gilt $d\eta = 0$ und somit auch $\vec{d\eta} = \vec{0}$.
Somit gilt für $\vec{ds}$ bzw. $ds$:
\begin{align}
	&\vec{ds} =  \vec{d\xi} \Rightarrow \nonumber \\
	&ds = \left|\left| \vec{d\xi} \right|\right| = \sqrt{ \left(\parDiff{x}{\xi}\right)^2 + \left(\parDiff{y}{\xi}\right)^2) d\xi} = \sqrt{ \left( \sum\limits_{k}\parDiff{N_k}{\xi}x_k \right)^2  + \left( \sum\limits_{k}\parDiff{N_k}{\xi}y_k \right)^2 } d\xi
\end{align}

Selbiges kann für Dreiecksseite 2 unter Verwendung von $d\xi = 0$ hergeleitet werden. Es ergibt sich somit:
\begin{align}
&\vec{ds} =  \vec{d\eta} \Rightarrow \nonumber \\
&ds = \left|\left| \vec{d\eta} \right|\right| = \sqrt{ \left(\parDiff{x}{\eta}\right)^2 + \left(\parDiff{y}{\eta}\right)^2) d\eta} = \sqrt{ \left( \sum\limits_{k}\parDiff{N_k}{\eta}x_k \right)^2  + \left( \sum\limits_{k}\parDiff{N_k}{\eta}y_k \right)^2 } d\eta
\end{align}

Zur Integration über Dreiecksseite 3  muss zuerst die Kurve parametrisiert werden:
\begin{align*}
&\vec{s(t)} = 
\begin{Bmatrix}
\xi(t) \\ \eta(t)
\end{Bmatrix} = 
\begin{Bmatrix}
1 - t \\ t
\end{Bmatrix} \Rightarrow\\
%
&\frac{ds}{dt} = 
\begin{Bmatrix}
\frac{d\xi}{dt} \\ \frac{d\eta}{dt}
\end{Bmatrix} = 
\begin{Bmatrix}
-1 \\ 1
\end{Bmatrix} \Rightarrow \\
% 
&\underline{\underline{d\xi = -dt, \ d\eta = dt}}
\end{align*}

Setzt man dies nun in (\ref{eq:subs}) und dies wiederum in (\ref{eq:ds_abs}) ein, so erhält man
\begin{equation}
ds = \sqrt{\left(\parDiff{x}{\eta} - \parDiff{x}{\xi}\right)^2  + \left(\parDiff{y}{\eta} - \parDiff{y}{\xi}\right)^2} dt
\end{equation}

Auf das Einsetzen der partiellen Ableitungen von $x$ und $y$ wurde aus Gründen der Übersichtlichkeit verzichtet.\newline


Die numerische Integration erfolgt wieder über die Gauss-Quadratur. Stützstellen und Gewichte wurden aus \cite{bathe}, S.542 übernommen.
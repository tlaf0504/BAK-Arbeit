\subsection{Problemtypen}
\label{sec:problem_types}
In diesem Kapitel werden die in der Software implementierten Problemtypen beschrieben. In beiden Fällen handelt es sich um zweidimensionale, ebene Probleme. Resultat dieses Abschnitts werden analytische Berechnungsvorschriften für die einzelnen Elementgleichungssysteme sein.\newline
Die Berechnungen aus diesem Abschnitt stammen, sofern nicht anders angegeben, aus \cite{SMS_VO_skript} Kapitel 7 und 8. Für ausführliche Herleitungen, auf welche in diesem Fall bewusst verzichtet wird, sei auf das oben genannte Werk verwiesen.\newline

Wie aus Abschnitt \ref{sec:fem_theory} bekannt, ist das Randwertproblem zuerst als Operatorgleichung mit entsprechenden Randwertbedingungen zu formulieren. Die Operatorgleichung kann nun entweder direkt im Galerkinschen Verfahren oder über Umweg eines äquivalenten Funktionals im Ritzschen Verfahren verwendet werden. Dabei wird für die gesuchte Funktion $u$ in jedem Teilgebiet (finites Element) ein entsprechender Approximations-Ansatz wie in Abschnitt \ref{sec:finite_elements_and_shape_functions} ermittelt, eingesetzt. Das Ergebnis sind 'lokale' lineare Gleichungssysteme (für jedes Element eines), welche zu einem großen, 'globalen' Gleichungssystem assembliert werden müssen. Auf die Assemblierung wird dabei genauer in Abschnitt \ref{sec:assembling} eingegangen.\newline

Ausgangspunkt für Probleme aus der Elektrotechnik sind im Allgemeinen die Maxwell-Gleichungen:
\begin{align}
	\mathit{rot}\vec{E} = -\parDiff{\vec{B}}{t} \label{eq:maxwell_1}, \\
	\mathit{rot}\vec{H} = \vec{J} + \parDiff{\vec{D}}{t} \label{eq:maxwell_2},\\
	\mathit{div}\vec{B} = 0 \label{eq:maxwell_3},\\
	\mathit{div}\vec{D} = \rho \label{eq:maxwell_4},
\end{align}
mit den Materialzusammenhängen
\begin{align}
\vec{D} = [\epsilon]\vec{E} \label{eq:material_1},\\
\vec{B} = [\mu]\vec{H} \label{eq:material_2},
\end{align}
und mit $[\epsilon]$ und $[\mu]$ als ortsabhängigen Materialtensoren.

\subsubsection{Elektrostatische Probleme}
\label{sec:electrostatic_problems}
Im elektrostatischen Fall gilt $\vec{J} \equiv 0,\ \vec{H} \equiv 0$, sowie für sämtliche Größen $\nicefrac{\partial}{\partial t}\equiv 0$.
Ein elektrostatisches Problem wird somit durch die Gleichungen
\begin{equation}
\mathit{rot}\vec{E} = 0 \label{eq:wirbelfreiheit_E}\\
\end{equation}
und
\begin{equation}
\mathit{div}\vec{D} = \rho \label{eq:sources_D}
\end{equation}

beschrieben.\newline

Setzt man (\ref{eq:e_grad_V}) nun unter Verwendung von (\ref{eq:material_1}) in (\ref{eq:sources_D}) ein, so erhält man als partielle Differentialgleichung für $V$ 
\begin{equation}
\mathit{div}[\epsilon]\mathit{grad}V = -\rho \ .
\end{equation}
Hierbei entspricht $\mathit{div}[\epsilon]\mathit{grad}$ dem Differentialoperator $L$ aus (\ref{eq:operatorgleichung}), das Potential $V$ der gesuchten Funktion $u$ und $-\rho$ der gegebenen Funktion $f$.\newline

Die Randbedingungen für ein solches Problem sind gegeben als
\begin{equation}
\label{eq:e-static_dirichlet_condition}
	V = \overline{V}
\end{equation}
am dirichletschen Rand $\Gamma_D$ und 
\begin{equation}
\label{eq:e-static_neumann_condition}
\vec{n}\cdot[\epsilon]\mathit{grad}V = \sigma 
\end{equation}
am neumannschen Rand $\Gamma_N$, mit $\vec{n}$ als Flächennormale und $\sigma$ als gegebener Flächenladungsdichte.\newline

Unter Verwendung des Ritzschen \textit{oder} Galerkinschen Verfahrens erhält man nun Formeln zur Berechnung der Elementgleichungssysteme, wobei beide Verfahren dieselben (!) Formeln liefern. Löst man (\ref{eq:first_variation}) oder (\ref{eq:weighted_residuum}) mit den entsprechenden Ansätzen, so ergibt sich für jedes Element ein quadratisches lineares \textit{Elementgleichungssystem} $\begin{bmatrix}k_{ij}\end{bmatrix} \cdot \begin{Bmatrix}V_j\end{Bmatrix} = \begin{Bmatrix}r_j\end{Bmatrix}$ der Dimension $N\times N$ mit $N$ als Anzahl der Elementknoten:
\begin{equation}
\label{eq:k_ij_e_static}
k_{ij} = \int\displaylimits_{x} \int\displaylimits_{y} \left(\epsilon_x \parDiff{N_i}{x}\parDiff{N_j}{x} +  \epsilon_y\parDiff{N_i}{y}\parDiff{N_j}{y}\right)dx dy,
\end{equation}


\begin{equation}
\label{eq:right_side_e_static}
	r_j = \int\displaylimits_{x} \int\displaylimits_{y} N_i \rho dx dy + \int\displaylimits_{\Gamma_N} N_i\sigma d\Gamma.
\end{equation}

\textbf{Anmerkung:} Um auf die oben gezeigte Form für $k_{ij}$ zu kommen ist es notwendig den Permettivitätstensor $[\epsilon]$ auf die Hauptachsenform
\begin{equation*}
	[\epsilon] = \begin{bmatrix}
	\epsilon_x & 0 \\
	0 & \epsilon_y
	\end{bmatrix}
\end{equation*}
zu transformieren.\newline

Man beachte dass für Formfunktionen von isoparametrischen finite Elementen $N_i = N_i(\xi, \eta)$ gilt, wodurch alle Integrale in den Variablen $\xi$ und $\eta$ durchgeführt werden müssen.


\subsubsection{Stationäre Strömungsfeldprobleme}
\label{sec:stat_current_problems}
Stationäre Strömungsfeldprobleme lassen sich über das Gesetz der Ladungserhaltung definieren:
\begin{equation}
\label{eq:charge_presistence}
	\mathit{div}\vec{J} = 0\ .
\end{equation}
Die Feldgröße $\vec{J}$ wird hierbei als die \textit{elektrische Stromdichte} bezeichnet. Sie ist über den, durch eine Fläche $\Gamma$ fließenden Strom $I$ durch folgenden Zusammenhang mit $\vec{n}$ als ortsabhängigem Normalvektor der Fläche definiert:
\begin{equation}
I = \int\displaylimits_{\Gamma} \vec{J} \cdot \vec{n} d\Gamma\ .
\end{equation}
Unter Verwendung von 
\begin{equation}
	U = \int\displaylimits_{c} \vec{E}\cdot \vec{ds}
\end{equation} kann das klassische Ohmsche Gesetz $U=RI$ auf seine differentielle Form
\begin{equation}
\label{eq:ohm_diff_form}
 	\vec{J} = [\gamma] \vec{E}
 \end{equation}
gebracht werden, wobei $[\gamma]$ den ortsabhängigen Tensor der spezifischen Leitfähigkeit darstellt.\newline
 
 Stationäre Strömungsfeldprobleme sind somit aufgrund von (\ref{eq:e_grad_V}), (\ref{eq:charge_presistence}) und (\ref{eq:ohm_diff_form}), analog zu elektrostatischen Problemen, durch folgende Differentialgleichung bestimmt:
 \begin{equation}
\mathit{div}[\gamma]\mathit{grad}V = 0 \ .
 \end{equation}
 Als Randbedingungen ergeben sich
 \begin{equation}
 \label{eq:current_dirichlet_condition}
 V = \overline{V}
 \end{equation}
 am dirichletschen Rand $\Gamma_D$ und 
 \begin{equation}
 \label{eq:current_neumann_condition}
 \vec{n}\cdot[\gamma]\mathit{grad}V = J_e
 \end{equation}
 am neumannschen Rand $\Gamma_N$, wobei $J_e$ die eingeprägte Flächenstromdichte am Rand darstellt.\newline
 
 Man erkennt die starken Äquivalenzen zwischen stationären Strömungsfeldproblemen und elektrostatischen Problemen. Das Elementgleichungssystem ergibt sich somit für diese Probleme als
 
\begin{equation}
\label{eq:k_ij_stat_current}
k_{ij} = \int\displaylimits_{x} \int\displaylimits_{y} \left(\gamma_x \parDiff{N_i}{x}\parDiff{N_j}{x} +  \gamma_y\parDiff{N_i}{y}\parDiff{N_j}{y}\right)dx dy,
\end{equation}

\begin{equation}
\label{eq:right_side_stat_current}
r_j = \int\displaylimits_{x} \int\displaylimits_{y} N_i J_e dx dy.
\end{equation}



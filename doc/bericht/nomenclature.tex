\newpage
\section*{Nomenklatur}

\subsection*{Allgemeine Formelzeichen}


\FloatBarrier
\begin{longtable}[l]{p{70pt} p{80pt} p{200pt}}
 \textbf{Symbol}	& \textbf{Einheit} & \textbf{Beschreibung} \\
 $\mathbf{H}$	& $\frac{A}{m}$	& Magnetische Erregung\\
 $\mathbf{J}$	& $\frac{A}{m^2}$	& Elektrische Stromdichte\\
 $\mathbf{B}$	& $\frac{Vs}{m^2}$	& Magnetische Flussdichte\\
 $\mathbf{E}$	& $\frac{V}{m}$	& Elektrische Feldst\"arke\\
 $\mathbf{D}$	& $\frac{As}{m^2}$	& Elektrische Flussdichte\\
 $\mathbf{A}$	& $\frac{Vs}{m}$	& Magnetisches Vektorpotential\\
 $\phi$	& $V$	& Elektrisches Skalarpotential\\
 $I$	& $A$	& elektrischer Strom\\
 $\rho$	& $\frac{As}{m^3}$	& Raumladungsdichte\\
 $\omega$	& $\frac{rad}{s}$	& Kreisfrequenz\\
 $f$	& $\frac{1}{s}$	& Frequenz\\
 $\lambda$	& $m$	& Wellenl\"ange\\
 $\Omega$	& $m^3$	& Volumen\\
 $\varepsilon$	& $\frac{As}{Vm}$	& Permittivit\"at\\
 $\mu$	& $\frac{Vs}{Am}$	& Permeabilit\"at
\end{longtable}
\FloatBarrier
\subsection*{Antennenspezifische Formelzeichen}
\FloatBarrier
\begin{longtable}[l]{p{70pt} p{80pt} p{80pt} p{200pt}}
 \textbf{Symbol}	& \textbf{Einheit}& \textbf{Definition} & \textbf{Beschreibung} \\
 $\beta$	& $\frac{1}{m}$	& $\frac{2\pi}{\lambda}$ & Phasenkonstante\\
 $D(\phi,\theta)$	& 1	&  & Richtcharakteristik\\
 $U(\phi,\theta)$	& $\frac{W}{4\pi sr}$	&  & Strahlungsintensität\\
 $\eta$	& $\Omega$	& $\sqrt{\frac{\mu_0}{\varepsilon_0}}$ & Feldwellenwiderstand des Vakuums
\end{longtable}
Die Antennenspezifischen Größen wurden an \cite{balanis} angelehnt, und können dort nachgeschlagen werden.
\FloatBarrier
\subsection*{Geometriespezifische Formelzeichen}


\FloatBarrier
\begin{longtable}[l]{p{70pt} p{80pt} p{200pt}}
 \textbf{Symbol}	& \textbf{Einheit} & \textbf{Beschreibung} \\
 $a$	& $m$	& Antennenradius\\
 $b$	& $m$	& Leiterradius der Antenne\\
 $U$	& $m$	& Antennenumfang\\
 $\mathbf{r'}$	& $m$	& Quellpunktsvektor\\
 $\mathbf{r}$	& $m$	& Aufpunktsvektor
\end{longtable}
\FloatBarrier
\newpage